\documentclass[letterpaper,11pt]{article}

\usepackage{latexsym}
\usepackage[empty]{fullpage}
\usepackage{titlesec}
\usepackage{marvosym}
\usepackage[usenames,dvipsnames]{color}
\usepackage{verbatim}
\usepackage{enumitem}
\usepackage[hidelinks]{hyperref}
\usepackage{fancyhdr}
\usepackage[english]{babel}
\usepackage{tabularx}
\usepackage{fontawesome5}
\usepackage{multicol}
\setlength{\multicolsep}{-3.0pt}
\setlength{\columnsep}{-1pt}
\input{glyphtounicode}

\pagestyle{fancy}
\fancyhf{}
\fancyfoot{}
\renewcommand{\headrulewidth}{0pt}
\renewcommand{\footrulewidth}{0pt}


\addtolength{\oddsidemargin}{-0.6in}
\addtolength{\evensidemargin}{-0.5in}
\addtolength{\textwidth}{1.19in}
\addtolength{\topmargin}{-.7in}
\addtolength{\textheight}{1.4in}

\urlstyle{same}

\raggedbottom
\raggedright
\setlength{\tabcolsep}{0in}


\titleformat{\section}{
  \vspace{-4pt}\scshape\raggedright\large\bfseries
}{}{0em}{}[\color{black}\titlerule \vspace{-5pt}]

\pdfgentounicode=1

% Custom commands
\newcommand{\resumeItem}[1]{
  \item\small{
    {#1 \vspace{-1pt}}
  }
}

\newcommand{\classesList}[4]{
    \item\small{
        {#1 #2 #3 #4 \vspace{-2pt}}
  }
}

\newcommand{\resumeSubheading}[4]{
  \vspace{-2pt}\item
    \begin{tabular*}{1.0\textwidth}[t]{l@{\extracolsep{\fill}}r}
      \textbf{#1} & \textbf{\small #2} \\
      \textit{\small#3} & \textit{\small #4} \\
    \end{tabular*}\vspace{-7pt}
}

\newcommand{\resumeSubSubheading}[2]{
    \item
    \begin{tabular*}{0.97\textwidth}{l@{\extracolsep{\fill}}r}
      \textit{\small#1} & \textit{\small #2} \\
    \end{tabular*}\vspace{-7pt}
}

\newcommand{\resumeProjectHeading}[2]{ 
    \item
    \begin{tabular*}{\textwidth}{@{}l} 
      \textbf{\small #1} | \textbf{\small #2} \\ 
    \end{tabular*}\vspace{-5pt}
}


\newcommand{\resumeSubItem}[2]{ 
  \item
    \begin{tabular*}{\textwidth}{@{}l@{}l}
      {\small\textbf{#1}} {\small #2} \\ 
    \end{tabular*}\vspace{-6pt}
}

\renewcommand\labelitemi{$\vcenter{\hbox{\tiny$\bullet$}}$}
\renewcommand\labelitemii{$\vcenter{\hbox{\tiny$\bullet$}}$}

\newcommand{\resumeSubHeadingListStart}{\begin{itemize}[leftmargin=0.0in, label={}]}
\newcommand{\resumeSubHeadingListEnd}{\end{itemize}}
\newcommand{\resumeItemListStart}{\begin{itemize}}
\newcommand{\resumeItemListEnd}{\end{itemize}\vspace{-5pt}}

%%%%%%  RESUME STARTS HERE  %%%%%%
\begin{document}

%---------------HEADER---------------
\begin{center}
  {\Huge \scshape Hospice Hounfodji} \\ \vspace{6pt}
  {\small \scshape Ingénieur IA} \\ \vspace{1pt}
  \small
  \href{mailto:hospicehounfodji@gmail.com}{\raisebox{-0.2\height}\faEnvelope\  \underline{hospicehounfodji@gmail.com}} ~
  \href{https://www.linkedin.com/in/hospice-hounfodji-44a2a7235/}{\raisebox{-0.1\height}\faLinkedin\ \underline{hospicehounfodji}} ~
  \href{https://github.com/hounfodji}{\raisebox{-0.1\height}\faGithub\ \underline{hounfodji}} ~
  % \href{https://hospice-dev.vercel.app/}{\raisebox{-0.2\height}\faGlobe\ \underline{hospicehounfodji.com}}
  \vspace{-8pt}
\end{center}

\vspace{1mm}
%---------------About---------------
\section{À propos}
\textbf{Ingénieur en intelligence artificielle}
{avec une année d'expérience dans le développement de modèles d'apprentissage automatique. Compétent en Python, Pytorch, Dart et traitement de données. Expérience dans la conception d'applications basées sur les modèles de fondation, ainsi que dans l'analyse et la visualisation de données. Passionné par l'innovation et la résolution de problèmes complexes à l'aide de l'IA.}

%---------------EDUCATION---------------
\section{Éducation}
\resumeSubHeadingListStart
\resumeSubheading
{Epitech Bénin}{MSc Pro Innovation - en cours}
{Intelligence artificielle}{2024 - 2026}

\resumeSubHeadingListEnd

\resumeSubHeadingListStart
\resumeSubheading
{Ecole Polytechnique d'Abomey-Calavi}{Ingénierie de Conception}
{Réseau Informatique et Internet}{2019 - 2025}
\resumeSubHeadingListEnd

\vspace{1mm}

%---------------EXPERIENCE---------------
\section{Expérience}
\resumeSubHeadingListStart

\resumeSubheading
{Alpilink}{Octobre 2024 -- Aujourd'hui}
{Ingénieur IA - Temps Partiel}{France}
\resumeItemListStart
\resumeItem{\textbf{Analyse de larges données} de fréquentation des stations de skis pour prédire les tendances de fréquentation et mesurer l'impact du dynamic pricing sur la fréquentation des stations de skis.}
\resumeItem{\textbf{Conception d'un assistant}, pour le BackOffice de vente de forfait de ski.}
\resumeItemListEnd

\resumeSubheading
{AIDA}{Mai 2024 -- Aujourd'hui}
{Développeur IA et mobile - Bénévolat}{Bénin}
\resumeItemListStart
\resumeItem{Développement de l'application mobile alodometo, Assistant Vocal Multifonctionnel, lauréate du 2ème prix au \textbf{Hackathon IA} challenge multimodal et multilingue - Bénin.}
\resumeItem{Transformation des maquettes de conception en une application mobile fonctionnelle en utilisant le framework \textbf{Flutter}.}
\resumeItemListEnd

\resumeSubheading
{iitech Bénin}{Août 2022 -- Octobre 2023}
{Développeur Web Fullstack - Stage}{Bénin}
\resumeItemListStart
\resumeItem{Migration des sites web de l'entreprise de Laravel 8 à Laravel 9, assurant une amélioration des performances et de la sécurité.}
\resumeItem{Collaboration avec des développeurs seniors dans un environnement d'équipe, participant à des revues de code et des sessions de partage de connaissances.}
\resumeItemListEnd

\resumeSubHeadingListEnd

%---------------Projects---------------
\section{Projets \footnotesize}
\resumeSubHeadingListStart

\resumeProjectHeading
{\textbf{{GrowGreen}} $|$ \emph{\href{https://github.com/hounfodji/watering_plants}{Source Code}}}{Flutter $|$ Firebase}
\\[5mm]
\resumeItemListStart
\resumeItem{Application mobile pour la gestion automatique de l'arrosage des plantes réalisée avec Flutter et Firebase}
\resumeItem{L'utilisateur peut lancer l'arrosage à distance, programmer des sessions automatiques et visualiser les données provenant des  capteurs.}
\resumeItemListEnd


\resumeProjectHeading
{\textbf{{AI Blog App}} $|$ \emph{\href{https://github.com/hounfodji/ai_blog_generator}{Source Code}}}{Django $|$ Gemini AI $|$ assemblyai}
\\[5mm]
\resumeItemListStart
\resumeItem{Génération automatique des articles de blog à partir de vidéos YouTube}
\resumeItem{Utilisation de l'IA Générative.}
\resumeItemListEnd



\resumeSubHeadingListEnd

%---------------Technial Skills---------------
\section{Compétences}
\resumeSubHeadingListStart
\resumeSubItem{Langages de programmation:}
{ Python, Dart, C/C++, SQL.}
\resumeSubItem{Outils:}
{Numpy, Pandas, Pytorch, Scikit-learn, Docker.}
\resumeSubItem{Techniques:}
{Apprentissage automatique, analyse et prétraitement des données.}
\resumeSubItem{Transversales:}
{Autonomie, communication, travail d'équipe.}
\resumeSubHeadingListEnd

%---------------Awards---------------
\section{Prix et distinctions}
\resumeSubHeadingListStart
\resumeSubItem{HACKATHON IA CHALLENGE MULTIMODAL ET MULTILINGUE - BÉNIN:}
{2ème, TEAM AIDA.}

\resumeSubHeadingListEnd


\end{document}