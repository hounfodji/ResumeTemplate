\documentclass[letterpaper,11pt]{article}

\usepackage{latexsym}
\usepackage[empty]{fullpage}
\usepackage{titlesec}
\usepackage{marvosym}
\usepackage[usenames,dvipsnames]{color}
\usepackage{verbatim}
\usepackage{enumitem}
\usepackage[hidelinks]{hyperref}
\usepackage{fancyhdr}
\usepackage[english]{babel}
\usepackage{tabularx}
\usepackage{fontawesome5}
\usepackage{multicol}
\setlength{\multicolsep}{-3.0pt}
\setlength{\columnsep}{-1pt}
\input{glyphtounicode}

\pagestyle{fancy}
\fancyhf{}
\fancyfoot{}
\renewcommand{\headrulewidth}{0pt}
\renewcommand{\footrulewidth}{0pt}


\addtolength{\oddsidemargin}{-0.6in}
\addtolength{\evensidemargin}{-0.5in}
\addtolength{\textwidth}{1.19in}
\addtolength{\topmargin}{-.7in}
\addtolength{\textheight}{1.4in}

\urlstyle{same}

\raggedbottom
\raggedright
\setlength{\tabcolsep}{0in}


\titleformat{\section}{
  \vspace{-4pt}\scshape\raggedright\large\bfseries
}{}{0em}{}[\color{black}\titlerule \vspace{-5pt}]

\pdfgentounicode=1

% Custom commands
\newcommand{\resumeItem}[1]{
  \item\small{
    {#1 \vspace{-1pt}}
  }
}

\newcommand{\classesList}[4]{
    \item\small{
        {#1 #2 #3 #4 \vspace{-2pt}}
  }
}

\newcommand{\resumeSubheading}[4]{
  \vspace{-2pt}\item
    \begin{tabular*}{1.0\textwidth}[t]{l@{\extracolsep{\fill}}r}
      \textbf{#1} & \textbf{\small #2} \\
      \textit{\small#3} & \textit{\small #4} \\
    \end{tabular*}\vspace{-7pt}
}

\newcommand{\resumeSubSubheading}[2]{
    \item
    \begin{tabular*}{0.97\textwidth}{l@{\extracolsep{\fill}}r}
      \textit{\small#1} & \textit{\small #2} \\
    \end{tabular*}\vspace{-7pt}
}

\newcommand{\resumeProjectHeading}[2]{ 
    \item
    \begin{tabular*}{\textwidth}{@{}l} 
      \textbf{\small #1} | \textbf{\small #2} \\ 
    \end{tabular*}\vspace{-5pt}
}


\newcommand{\resumeSubItem}[2]{ 
  \item
    \begin{tabular*}{\textwidth}{@{}l@{}l}
      {\small\textbf{#1}} {\small #2} \\ 
    \end{tabular*}\vspace{-6pt}
}

\renewcommand\labelitemi{$\vcenter{\hbox{\tiny$\bullet$}}$}
\renewcommand\labelitemii{$\vcenter{\hbox{\tiny$\bullet$}}$}

\newcommand{\resumeSubHeadingListStart}{\begin{itemize}[leftmargin=0.0in, label={}]}
\newcommand{\resumeSubHeadingListEnd}{\end{itemize}}
\newcommand{\resumeItemListStart}{\begin{itemize}}
\newcommand{\resumeItemListEnd}{\end{itemize}\vspace{-5pt}}

%%%%%%  RESUME STARTS HERE  %%%%%%
\begin{document}

%---------------HEADER---------------
\begin{center}
  {\Huge \scshape Hospice Hounfodji} \\ \vspace{6pt}
  {\small \scshape Software Developer} \\ \vspace{1pt}
  \small
  \href{mailto:hospicehounfodji@gmail.com}{\raisebox{-0.2\height}\faEnvelope\  \underline{hospicehounfodji@gmail.com}} ~
  \href{https://www.linkedin.com/in/hospice-hounfodji-44a2a7235/}{\raisebox{-0.1\height}\faLinkedin\ \underline{LinkedIn}} ~
  \href{https://github.com/hounfodji}{\raisebox{-0.1\height}\faGithub\ \underline{hounfodji}} ~
  \href{https://hospice-dev.vercel.app/}{\raisebox{-0.2\height}\faGlobe\ \underline{hospice-dev.vercel.app}}
  \vspace{-8pt}
\end{center}

%---------------EDUCATION---------------
\section{Education}
\resumeSubHeadingListStart
\resumeSubheading
{Ecole Polytechnique d'Abomey-Calavi}{Diplome Mars 2025}
{Réseau Informatique et Internet}{Abomey-Calavi}
\resumeItemListStart
\resumeItem{Cours: Administration systeme et réseaux, sécurité systeme et réseaux informatiques, architecture des
applications Web et Web dynamique, programmation Web avancée, data warehouse et data mining, programmation
orientée objet avancée, systemes embarqués temps réel, théorie des langages et traducteurs, initiation a l'IA, création et
gestion d'entreprise. }
\resumeItemListEnd
\resumeSubHeadingListEnd

%---------------EXPERIENCE---------------
\section{Experience}
\resumeSubHeadingListStart

\resumeSubheading
{BLUE LIFE TECH}{Août 2023 -- Septembre 2023}
{Stagiaire Développeur Web Fullstack}{Sègbèya-Sud, Cotonou}
\resumeItemListStart
\resumeItem{Développement d'un blog d'histoire et d'un site e-commerce avec ReactJS et Django. }
\resumeItem{Initiation à la méthode Scrum.}
\resumeItemListEnd

\resumeSubheading
{iitech Bénin}{Août 2022 -- Octobre 2023}
{Stagiaire Développeur Web Fullstack}{Abomey Calavi}
\resumeItemListStart
\resumeItem{Conception d'interfaces utilisateur intuitives, sélectionnant des modèles adaptés pour optimiser l'expérience utilisateur.}
\resumeItem{Création de designs responsifs avec HTML, CSS et Bootstrap, améliorant l'esthétique et les performances des sites.}
\resumeItem{Utilisation de Laravel pour le backend.}
\resumeItemListEnd

\resumeSubHeadingListEnd

%---------------Projects---------------
\section{Projets \footnotesize}
\resumeSubHeadingListStart

\resumeProjectHeading
{\textbf{{alo Do me to Assistant Vocal Multifonctionnel}} $|$ \emph{\href{https://github.com/hounfodji/AIDA}{Source Code}}}{Flutter $|$ Python $|$ Gemini AI}
\\[5mm]
\resumeItemListStart
\resumeItem{Application mobile innovante offrant des fonctionnalités de pointe. Les technologies utilisées incluent des modèles de langue pour la conversion speech to text au text to text, ainsi que Flutter pour le développement de l'application mobile.}
\resumeItemListEnd

\resumeProjectHeading
{\textbf{{Robust API}} $|$ \emph{\href{https://github.com/hounfodji/robust_api}{Source Code}}}{Django $|$ DRF $|$ Flutter $|$ Docker}
\\[5mm]
\resumeItemListStart
\resumeItem{Développement d'une API robuste avec Django et Django Rest Framework, suivant les principes de la méthodologie \textbf{12 Factor App}. L'application Flutter associée utilise des technologies modernes comme Riverpod, Freezed, et Dio pour une architecture propre et efficace.}
\resumeItem{Implémentation de fonctionnalités avancées incluant l'authentification, la sérialisation, et la gestion de modèles de données. Utilisation de Docker pour le déploiement et de tests automatisés pour assurer la qualité du code.}
\resumeItemListEnd

\resumeProjectHeading
{\textbf{{Smart Watering Plants}} $|$ \emph{\href{https://github.com/hounfodji/watering_plants}{Source Code}}}{Flutter $|$ Firebase}
\\[5mm]
\resumeItemListStart
\resumeItem{Application mobile pour la gestion automatique de l'arrosage des plantes réalisée avec Flutter et Firebase}
\resumeItem{L'utilisateur peut lancer l'arrosage à distance, programmer des sessions automatiques et visualiser les données provenant des  capteurs.}
\resumeItemListEnd


\resumeProjectHeading
{\textbf{{AI Blog App}} $|$ \emph{\href{https://github.com/hounfodji/ai_blog_generator}{Source Code}}}{Django $|$ Gemini AI $|$ assemblyai}
\\[5mm]
\resumeItemListStart
\resumeItem{Génération automatique des articles de blog à partir de vidéos YouTube}
\resumeItem{Utilisation de l'IA Générative.}
\resumeItemListEnd



\resumeSubHeadingListEnd

%---------------Technial Skills---------------
\section{Compétences techniques}
\resumeSubHeadingListStart
\resumeSubItem{Langages:}
{ Python, Javascript, Java, PHP, SQL, Dart, Kotlin, HTML5, CSS.}
\resumeSubItem{Technologies et Frameworks:}
{ReactJS, Django, Laravel, Flutter, Docker.}
\resumeSubItem{Databases:}
{MySQL, MongoDB, PostgreSQL.}
\resumeSubItem{Outils de développement:}
{ Linux, Git/Github, Firebase Hosting, Postman.}
\resumeSubHeadingListEnd

%---------------Awards---------------
\section{Prix et distinctions}
\resumeSubHeadingListStart
\resumeSubItem{HACKATHON IA CHALLENGE MULTIMODAL ET MULTILINGUE - BÉNIN:}
{2ème, TEAM AIDA.}

\resumeSubHeadingListEnd


\end{document}